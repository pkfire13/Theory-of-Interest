\documentclass[12pt]{article}
\usepackage{amsmath}
\usepackage{tikz}
\usepackage{actuarialangle}
\begin{document}
	
\title{Ch1 and Ch2 Concepts and Definitions}
\author{Paul Kim}
\date{\today}
\maketitle

\section{CH1}
\subsection{Effective Interest Rate and Discount Rate}
Definitions
\begin{enumerate}
	\item amount of interests from $X_1= (X_2 - X_1)$ 
	\item effective interest rate for time $[t_1, t_2]$ = $i_{[t_1,t_2]}$ = $\frac{X_2 - X_1}{X_1}$ 
	\item amount of discount from X$_2 = (X_2 - X_1)$ 
	\item effective discount rate for time $[t_1, t_2]$ = $d_{[t_1,t_2]}$ = $\frac{X_2 - X_1}{X_2}$ 
\end{enumerate}
$(1+i)(1-d) = 1$ for compound interest
\subsection{Accumulation Functions}
Definitions
\begin{enumerate}
	\item amount function: $A_K(t)$ for principal K dollars
	\item accumulation function $a(t)$ 
\end{enumerate}
note: $A_K(t) = Ka(t)$
\begin{enumerate}
	\item simple interest accumulation function at rate s: $a(t) = (1+st)$
	\item compound interest accumulation function at rate i: $a(t) = (1+i)^t$
	\item simple discount accumulation function at rate d: $a(t) = \frac{1}{(1-dt)}$
	\item compound discount accumulation function at rate d: $a(t) = \frac{1}{(1-d)^t}$
\end{enumerate}

n-th time period: n-1,n

\subsection{Discount Function, Discount Factor, PV, NPV}
Definitions
\begin{enumerate}
	\item discount function: $v(t) = \frac{1}{a(t)}$
	\item discount factor: $v = \frac{1}{1+i}$
	\item PV = present value = value at time 0
	\item NPV = net present value 
\end{enumerate}
the accumulated value of X (given at t1) at t2 = Xv(t1)a(t2)

\subsection{Nominal Interest, Nominal Discount}

\begin{enumerate}
	\item nominal interest rate convertible m times per year = $i^{(m)}$
	\item effective m-th interest rate = $i^{(m)}/m$
\end{enumerate}
\begin{enumerate}
	\item nominal discount rate convertible m times per year = $d^{(m)}$
	\item effective m-th discount rate = $d^{(m)}/m$
\end{enumerate}
time rates
\begin{enumerate}
	\item annually = m = 1
	\item semiannually = m =2
	\item quarterly = m = 4
	\item monthly = m = 12
\end{enumerate}
biannual = every two years; m = 1/2
"convertible, payable or compounded"
converting interest rate
\begin{align}
	(1+i_{annual}) = (1+i_{m})^{m} \nonumber
\end{align}

\subsection{Constant force of interest and force of interest function}
\begin{enumerate}
	\item \#\% constant force of interest ($\delta = \#\%$) = $i = e^{\delta} - 1$
	\item force of interest is constant, grows at a rate of $\delta$ compounded continuously
	\item compounded continuously at a rate of $\delta$; a(t) = $e^{\delta t}$
\end{enumerate}
\begin{enumerate}
	\item force of interest function: $\delta_t = \frac{a'(t)}{a(t)} = \frac{d}{dt}(ln*a(t))$
\end{enumerate}

\subsection{Inflation}
\begin{enumerate}
	\item inflation adjusted interest rate = j = purchase power percentage = real interest rate 
	\item non-inflation adjusted interest rate; stated interest rate = i
	\item inflation rate = r 
\end{enumerate}
\begin{align}
	1 + j_{[t_1,t_2]} = \frac{1 + i_{[t_1, t_2]}}{1 + r_{[t_1,t_2]}} \nonumber
\end{align}

\section{CH2}
\subsection{Yield and Time-Weighted Yield}
\begin{enumerate}
	\item yield rate; internal rate of return; dollar-weighted yield rate
	\item time-weighted yield
\end{enumerate}
\end{document}