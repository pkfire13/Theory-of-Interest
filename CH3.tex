\documentclass[12pt]{article}
\usepackage{amsmath}
\usepackage{tikz}
\usepackage{actuarialangle}
\begin{document}
	
\title{CH3 Concepts and Definitions}
\author{Paul Kim}
\date{\today}
\maketitle

\section{CH3}
\subsection{Annuities Basic}
\begin{align}
	a + ar + ar^2 + ... + ar^{n-1} = a \frac{1-r^n}{1 - r} \nonumber
\end{align}
\begin{enumerate}
	\item immediate = end of each payment period
	\begin{enumerate}
		\item $a_{\angl{n}i}$ and $s_{\angl{n}i}$
		\item "END"
	\end{enumerate}

	\item due = beginning of each payment period
	\begin{enumerate}
		\item $\ddot{a}_{\angl{n}i}$ and $\ddot{s}_{\angl{n}i}$
		\item "BEGIN"
	\end{enumerate}

\end{enumerate}
\begin{tikzpicture}
	% draw horizontal line
	\draw (0,0) -- (12,0);
	
	%draw vertical lines
	\foreach \x in {0,4,8,12}
	\draw (\x cm, 3pt) -- (\x cm, -3pt);
	
	%draw nodes
	
	\draw (0,0) node[below=3pt] {$a_{\angl{n}i}$} node[above=3pt] {t=0};
	
	\draw (4,0) node[below=3pt] {PMT \#1} node[below=18pt] {$\ddot{a}_{\angl{n}i}$} node[above=3pt] {1};
	
	\draw (8,0) node[below=3pt] {PMT \#2} node[below=18pt] {$s_{\angl{n}i}$} node[above=3pt] {2};
	
	\draw (12,0) node[below=3pt] {$\ddot{s}_{\angl{n}i}$} node[above=3pt] {3};
\end{tikzpicture}

\subsection{loans with sightly reduced final payment}
\begin{enumerate}
	\item calculate for pmt
	\item recalculate pv with pmt
	\item if the pv is greater than loan value , go to 5
	\item if the pv is less than loan value, round up the pmt
	\item calculate the value of loan at the end period
	\item calculate the value of the annuity at the end of the annuity (s)
	\item subtract with \#6 and \#5 in that order (6 should be higher now)
	\item subtract the delta with the pmt and that's your last reduced pmt
\end{enumerate}
\subsection{perpetuities and dividend model}
\begin{align}
	a_{\angl{\infty}i} = \frac{1}{i} \nonumber \\
	 \ddot{a}_{\angl{\infty}i} = \frac{1}{d} \nonumber
\end{align}
\begin{enumerate}
	\item 
\end{enumerate}

\subsection{Outstanding Loan Balance}
\begin{enumerate}
	\item $OLB_k$ = outstanding loan balance right after the k-th pmt
	\begin{enumerate}
		\item Q$a_{\angl{k}i}$
	\end{enumerate}
	\item prospective: (total PV value of all remaining pmts)
	\item retrospective: (Value of the loan at time k) - (total FV value of payments made)
	\begin{enumerate}
		\item loan value*$(1+i)^k - Qs_{\angl{k}i}$
	\end{enumerate}
\end{enumerate}
\subsection{Non-leveled Annuities}
\begin{enumerate}
	\item ?
\end{enumerate}

\subsection{Annuities with Geometric progression pmts}
\begin{align}
	a + ar + ar^2 + ... + ar^{n-1} = a \frac{1-r^n}{1 - r} \nonumber \\
	a + ar + ar^2 + ... = a \frac{1}{1-r} \nonumber
\end{align}

\subsection{with Arithmetic progression pmts}
P, P+Q, P+2Q, ... P+(n-1)Q where P is the pmt \\
v = $\frac{1}{1+i}$ 
\begin{align}
	(I_{P,Q}a)_{\angl{n}i} = Pa_{\angl{n}i} + \frac{Q}{i}(a_{\angl{n}i} - nv^n) \nonumber \\	
	(I_{P,Q}s)_{\angl{n}i} = Ps_{\angl{n}i} + \frac{Q}{i}(s_{\angl{n}i} - n) \nonumber \\
	(I_{P,Q}\ddot{a})_{\angl{n}i} = P\ddot{a}_{\angl{n}i} + \frac{Q}{i}(a_{\angl{n}i} - nv^n) \nonumber \\	
	(I_{P,Q}\ddot{s})_{\angl{n}i} = P\ddot{s}_{\angl{n}i} + \frac{Q}{i}(s_{\angl{n}i} - n) \nonumber 
\end{align}
\begin{align}
	(I_{P,Q}a)_{\angl{\infty}i} = \frac{P}{i} + \frac{Q}{i^2} \nonumber \\		
	(I_{P,Q}\ddot{a})_{\angl{\infty}i} = \frac{P}{d} + \frac{Q}{id} \nonumber
\end{align}

\subsection{Annuity Paid Continuously}
\begin{enumerate}
	\item value at t0 of the level annuity with continuous payment to t = n, at a rate of 1 for each period of length = $\int_{0}^{n} v dt$
\end{enumerate}
if $a(t) = (1+i)^t$
\begin{align}
	\overline{a}_{\angl{n}i} = \frac{1-v^n}{\delta} = \frac{1-v^n}{ln(1+i)} \nonumber \\
	\overline{s}_{\angl{n}i} = \frac{(1+i)^n - 1}{\delta} = \frac{(1+i)^n - 1}{ln(1+i)} \nonumber
\end{align}
\begin{enumerate}
	\item annuity paid continuously at a rate of f(t)
	\item amount paid during time [a,b] = $\int_{a}^{b} f(t)dt$
	\item amount paid during period [t, t+dt] = f(t)dt
	\item value at t0 = $\int_{0}^{n}v(t)f(t)dt$
	\item "annuity paid continuously at constant rate of 1": f(t) = 1
	\item "interest is paid continuously at a rate of $\delta$": a(t) = $e^{\delta t }$
\end{enumerate}

\end{document}